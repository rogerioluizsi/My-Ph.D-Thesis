Building on the insights from the previous chapters, which demonstrated the utility and reliability of ALE in clarifying the roles of features in models performing well amidst non-independent data, this chapter introduces a case study employing the proposed ALE-based metrics. These metrics are applied to assess the impact of various features on student performance, as well as to trace the evolution of these influences over time. 

The adoption of scores offers a more apt alternative compared to other explanation types, especially in the context of analyzing feature effects over an extended period within complex multivariate scenarios \cite{SilvaFilho2023AAchievement}. Through this reporting approach, educational practitioners can gain a clearer understanding of how numerous features fluctuate over time, thereby improving the interpretation of models to support data-driven decision-making.

Scores provide a succinct, overarching measure of the model's output, distilling the essence of complex relationships into a single, interpretable metric. This approach is particularly advantageous when the primary goal is to gain an initial, high-level understanding of the model's behavior across multiple dimensions. While scores may not offer the nuanced details of feature-specific effects, they serve as an effective starting point from a human-centric perspective for further, in-depth analysis. Specifically, in this study case, using a unique score to represent the overall contribution of features facilitates a more transparent and comprehensive understanding of the role of many features over time.

\section{Introduction}

A prominent application of educational data mining is in exploring large-scale educational assessments (LSAs). The advent of machine learning (ML) as an alternative to traditional statistical models, which have been prevalent in policy-oriented research since Coleman's 1968 study \cite{coleman1968equality}, marks a significant shift in this domain.

The modernization of LSAs has notably improved data collection regarding educational system performance\cite{Hernandez-Torrano2021ModernLiteratureb}. Beyond performance metrics, LSAs gather information on the education system, including students' socio-demographic and school characteristics. A prime example is the Programme for International Student Assessment (PISA), which offers a global perspective on secondary education learning outcomes \cite{VarkeyFoundation2018GlobalSTATISTICS}. Following international efforts, national LSAs have also played a significant role in the evaluation and improvement of their educational systems \cite{Johansson2016InternationalConsequences}.

The availability of vast, structured educational data has spurred researchers' interest in more flexible methods that overcome the limitations of traditional statistical techniques. These limitations include constraints in handling a large number of variables without prior assumptions about the data \cite{Martinez-Abad2018BigEducation, Masci2018StudentApproach}. These studies often utilize the supervised learning paradigm to predict LSA achievement (output) based on contextual variables from LSA questionnaires (inputs). Characterizing this input-output mapping enables the identification of sources for educational policies and practices associated with academic achievement. This characterization has been expressed either through the models themselves \cite{Gomes2020PresentingDataset, SilvaFilho2019DataInstitutes} or through post-hoc explanation techniques \cite{Gabriel2018ALiteracy, Schiltz2018UsingApproach}.

This chapter contributes to such research, aiming to identify and track key features related to student performance. The primary goal is to demonstrate the applicability of ALE-based metrics introduced in the previous chapter. Additionally, this chapter offers new contributions: 1) It defines a model-agnostic process for reporting trends in the predictive contribution of features in repeated cross-sectional data, and 2) it presents an innovative case study within the context of the Brazilian secondary education system, providing new insights for educational literature.



