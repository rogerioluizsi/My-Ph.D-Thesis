\subsection{Real-World Data}
\label{sec:realWolrldData}

To demonstrate the practical use of the new metrics in real-world data, two openly available scientific data from the UC Irvine (UCI) machine learning repository were used \footnote{\url{https://archive.ics.uci.edu/datasets}}. The breast cancer data \cite{Dua:2019} and an educational data \cite{Ylmaz2020StudentTechniques}. In these experiments, the focus is gaining a qualitative understanding of how differently the proposed and baseline metrics isolated and distinguished the effects of variables. 

The breast cancer data included benign and malignant cell samples from 369 patients, 212 with cancer, and 157 with fibrocystic breast masses. Each sample contained thirty features which were used by the model to predict the type of cancer. The educational data consisted of the performance of 145 higher education students, 57 with scores above or equal to 4 and 88 below, described in 31 variables linked to socioeconomics, demographic, and student behavior. At all, 30 features (all features less "STUDENT ID", "COURSE ID) were used to predict student grade. The grade target variable was transformed to a classification task and grades below 4, receive 0 otherwise 1.



